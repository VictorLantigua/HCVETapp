\chapter{Producto}\label{chapter:proposal}

Diseñada con el principal objetivo de crear y gestionar de manera sencilla e intuitiva historias clínicas veterinarias digitales de las mascotas en casa, la aplicación \textbf{HCVet} funciona tanto de forma local independiente ($offline$) como con conexión a un servidor utilizando internet. Teniendo como “único inconveniente” que se necesite una vez instalada la app en el móvil, conexión a internet para efectuar el registro en el servidor, la app fue pensada para que en todo momento el usuario tuviera el control total de los datos de sus mascotas. A través de diferentes vistas para distintos tipos de consulta el usuario puede introducir los datos y visualizarlos cuando desee, así, como eliminarlos cuando crea conveniente. Para esto, la app replica en una base de datos local los datos de sus mascotas que se sincronizan con los del servidor cuando exista una conexión a internet. La aplicación también permite el compartir mascotas vía wifi con otras personas que tengan instalada la aplicación a las que llamaremos “encargados”. Los encargados pueden almacenar en su base de datos local, y por ende visualizar y crear nuevas consultas, las mascotas a las cuales el dueño le ha compartido. Todo este proceso de sincronización se hará de forma automática o manual utilizando la conexión al servidor. Cabe señalar que una vez que el usuario se registre en el servidor podrá crear dos mascotas de forma gratuita, en caso de necesitar más deberá realizar una suscripción.

Entre las principales funcionalidades que ofrece la app se encuentran:

\begin{itemize}
\item \textbf{ Creación de una mascota}: el usuario podrá crear una nueva mascota en la aplicación; para esto tendrá que llenar los campos de información básica de una mascota entre los que se encuentran: nombre de la mascota, fecha de nacimiento, raza, especie, tipo de sangre, entre otros.
\item \textbf{ Eliminar mascota}: el usuario podrá eliminar una mascota de su aplicación. Cabe señalar que este es un proceso peligroso puesto que una vez realizado no se podrán recuperar los datos de la mascota eliminada. Si la mascota es propia se elimina tanto de su base de datos local como la que está almacenada en el servidor, en caso de ser una mascota que no es suya, solo se eliminará de su base de datos local y se notificará al servidor que ya no es encargado de dicha mascota.
\item \textbf{ Compartir/Recibir mascota}: el usuario podrá compartir/recibir los datos de una mascota (incluyendo todas las consultas de esta mascota) vía wifi. Una vez que exista conexión a internet se le notificará al servidor que el usuario que recibió la mascota ahora es un encargado de esta y por ende, el servidor se encargará de mantener sincronizados los datos de dicha mascota en ambos teléfonos (en el del dueño y en el del encargado).
\item \textbf{ Insertar nueva consulta}: el usuario podrá rellenar los campos de información para la creación de una nueva consulta. Entre las posibles consultas que ofrece la aplicación se encuentran las siguientes: Visitas Médicas, Pruebas de Laboratorio, Radiologías, Patologías, Cirugías y Prescripciones.
\item \textbf{ Insertar nueva Alergia/Condición/Vacuna}: el usuario podrá también llevar en la aplicación un registro de las vacunas, las alergias y las condiciones o enfermedades que se han realizado o diagnosticado a sus mascotas en dependencia del caso.
\item \textbf{ Insertar notas extras}: el usuario podrá insertar notas extras sobre las mascotas.
\item \textbf{ Visualización}: el usuario podrá visualizar de manera sencilla y organizada, ordenando por fecha, cada una de las consultas, así como las alergias, las vacunas, las condiciones, las notas extras, y los datos básicos de las mascotas. Teniendo la posibilidad de filtrar en cada caso.
\end{itemize}


